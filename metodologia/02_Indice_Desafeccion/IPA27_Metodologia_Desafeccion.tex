\documentclass[mathserif, utf8, aspectratio=169]{beamer}

% --- Temas y Estetica ---
\usetheme{Madrid}
\usecolortheme{default}
\useinnertheme{rounded}
\useoutertheme{shadow}

% --- Paquetes ---
\usepackage[spanish]{babel}
\usepackage{graphicx}
\usepackage{bookmark}
\usepackage{xcolor}
\usepackage{tcolorbox}
\tcbuselibrary{skins}
\usepackage{tikz}

% --- Configuracion de Colores ---
\definecolor{AndaluciaGreen}{RGB}{0, 107, 63}
\definecolor{VibrantOrange}{RGB}{255, 127, 0}
\definecolor{DeepBlue}{RGB}{0, 75, 147}

\setbeamercolor{palette primary}{bg=AndaluciaGreen, fg=white}
\setbeamercolor{palette secondary}{bg=DeepBlue, fg=white}
\setbeamercolor{palette tertiary}{bg=DeepBlue, fg=white}
\setbeamercolor{title}{bg=AndaluciaGreen, fg=white}
\setbeamercolor{frametitle}{bg=AndaluciaGreen, fg=white}
\setbeamercolor{block title}{bg=DeepBlue, fg=white}
\setbeamercolor{item}{fg=VibrantOrange}

% --- Configuracion de Cajas ---
\tcbset{
    enhanced,
    colback=white,
    colframe=DeepBlue,
    arc=3mm,
    fonttitle=\bfseries,
    boxrule=1pt
}

% --- Metadatos ---
\title[IPA27 - Desafeccion Politica]{Indicator: Indice de Desafeccion (GOB\_DES)}
\subtitle{Metodologia de analisis de sentimientos y percepcion de problemas (CIS)}
\author[Proyecto IPA27]{Analisis Tecnico de Indicadores}
\date{\today}

\begin{document}

% --- Portada ---
{
\setbeamercolor{background canvas}{bg=AndaluciaGreen}
\begin{frame}[plain]
    \begin{center}
        \vspace{1cm}
        \textcolor{white}{\Huge \textbf{Indice de Desafeccion Politica}} \\
        \vspace{0.3cm}
        \textcolor{white}{\Large Cuantificando el descontento institucional} \\
        \vspace{1cm}
        \begin{tikzpicture}
            \fill[VibrantOrange] (0,0) circle (0.4cm) node[white] {\textbf{D}};
            \draw[white, line width=2pt] (0,0) circle (0.4cm);
        \end{tikzpicture} \\
        \vspace{1cm}
        \textcolor{white}{\large Fuente: Barometros CIS (PESPANNA1/2/3)} \\
    \end{center}
\end{frame}
}

% --- Slide 2: El Concepto ---
\begin{frame}{¿Como medimos la desafeccion?}
    \begin{columns}
        \begin{column}{0.6\textwidth}
            \begin{tcolorbox}[title=Definicion Operativa]
                La desafeccion se mide indirectamente a traves de la **importancia relativa** que los ciudadanos otorgan a los problemas politicos en su agenda personal.
            \end{tcolorbox}
            \begin{itemize}
                \item No es solo "opinar mal", es considerar que **la politica en si misma es un problema**.
                \item Se basa en las tres menciones de "Principales problemas de España".
                \item Captura la fatiga institucional y la desconfianza.
            \end{itemize}
        \end{column}
        \begin{column}{0.4\textwidth}
            \centering
            \begin{tikzpicture}
                \node[draw=VibrantOrange, circle, inner sep=5pt] (P) at (0,0) {Politicos};
                \node[draw=AndaluciaGreen, circle, inner sep=5pt] (M) at (2,0) {Materiales};
                \draw[<->, line width=1.5pt] (P) -- (M) node[midway, above] {Crowding-out};
            \end{tikzpicture}
        \end{column}
    \end{columns}
\end{frame}

% --- Slide 3: Codigos de Identificacion ---
\begin{frame}{Identificando el "Problema Politico"}
    \begin{tcolorbox}[colframe=AndaluciaGreen, title=Foco en Desafeccion (Score Rojo)]
        Filtramos menciones especificas que reflejan descontento estructural:
    \end{tcolorbox}
    \begin{columns}
        \begin{column}{0.5\textwidth}
            \begin{itemize}
                \item \textbf{Cod. 11}: Corrupcion y fraude.
                \item \textbf{Cod. 13}: Mal comportamiento de los politicos.
                \item \textbf{Cod. 51}: Problemas politicos en general.
            \end{itemize}
        \end{column}
        \begin{column}{0.5\textwidth}
            \begin{itemize}
                \item \textbf{Cod. 24}: El Gobierno.
                \item \textbf{Cod. 50}: Partidos politicos.
            \end{itemize}
        \end{column}
    \end{columns}
    \vspace{0.3cm}
    \begin{small}
    *Estos codigos se comparan con problemas "Materiales" (Paro, Sanidad, Vivienda) para entender la competencia en la agenda ciudadana.
    \end{small}
\end{frame}

% --- Slide 4: Algoritmo de Calculo ---
\begin{frame}{Algoritmo de Ponderacion}
    \begin{columns}
        \begin{column}{0.5\textwidth}
            \begin{block}{1. Jerarquia de Mencíon}
                No todos los problemas pesan igual:
                \begin{itemize}
                    \item \textbf{1er Problema}: Peso 3.
                    \item \textbf{2do Problema}: Peso 2.
                    \item \textbf{3er Problema}: Peso 1.
                \end{itemize}
            \end{block}
        \end{column}
        \begin{column}{0.5\textwidth}
            \begin{block}{2. Normalización (0-100)}
                Se suma el score (max. 6 puntos) y se escala a centiles.
            \end{block}
            \begin{block}{3. Agregación Regional}
                Calculo de la media para cada CCAA usando los pesos muestrales del CIS.
            \end{block}
        \end{column}
    \end{columns}
\end{frame}

% --- Slide 5: El Factor de "Crowding-Out" ---
\begin{frame}{Desafeccion vs Realidad Material}
    \begin{tcolorbox}[title=Ajuste por Urgencia Social, colframe=VibrantOrange]
        El indice final tiene en cuenta si la desafeccion sube porque **realmente hay mas descontento** o porque **otros problemas (como el paro) han bajado**.
    \end{tcolorbox}
    \begin{itemize}
        \item \textbf{Analisis Factorial}: Se limpian sesgos mediante tecnicas de factorizacion para aislar la "desafeccion pura".
        \item \textbf{Sensibilidad}: El indicador es altamente sensible a ciclos electorales y crisis de gobernanza.
        \item \textbf{Aplicacion}: Clave en la dimension de **Sociedades Inclusivas (Pillar 3)**.
    \end{itemize}
\end{frame}

% --- Slide 6: Cierre ---
{
\setbeamercolor{background canvas}{bg=DeepBlue}
\begin{frame}[plain]
    \begin{center}
        \vspace{2cm}
        \textcolor{white}{\Huge \textbf{Escuchando el Pulso de la Ciudadania}} \\
        \vspace{1.5cm}
        \textcolor{VibrantOrange}{\large Metodologia Desafeccion IPA27} \\
    \end{center}
\end{frame}
}

\end{document}
