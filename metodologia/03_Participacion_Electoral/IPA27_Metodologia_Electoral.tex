\documentclass[mathserif, utf8, aspectratio=169]{beamer}

% --- Temas y Estetica ---
\usetheme{Madrid}
\usecolortheme{default}
\useinnertheme{rounded}
\useoutertheme{shadow}

% --- Paquetes ---
\usepackage[spanish]{babel}
\usepackage{graphicx}
\usepackage{bookmark}
\usepackage{xcolor}
\usepackage{tcolorbox}
\tcbuselibrary{skins}
\usepackage{tikz}

% --- Configuracion de Colores ---
\definecolor{AndaluciaGreen}{RGB}{0, 107, 63}
\definecolor{VibrantOrange}{RGB}{255, 127, 0}
\definecolor{DeepBlue}{RGB}{0, 75, 147}

\setbeamercolor{palette primary}{bg=AndaluciaGreen, fg=white}
\setbeamercolor{palette secondary}{bg=DeepBlue, fg=white}
\setbeamercolor{palette tertiary}{bg=DeepBlue, fg=white}
\setbeamercolor{title}{bg=AndaluciaGreen, fg=white}
\setbeamercolor{frametitle}{bg=AndaluciaGreen, fg=white}
\setbeamercolor{block title}{bg=DeepBlue, fg=white}
\setbeamercolor{item}{fg=VibrantOrange}

% --- Configuracion de Cajas ---
\tcbset{
    enhanced,
    colback=white,
    colframe=DeepBlue,
    arc=3mm,
    fonttitle=\bfseries,
    boxrule=1pt
}

% --- Metadatos ---
\title[IPA27 - Participacion Electoral]{Indicator: Participacion Electoral (SOC\_PAR\_enlazado)}
\subtitle{Metodologia de construccion a partir de microdatos del CIS}
\author[Proyecto IPA27]{Analisis Tecnico de Indicadores}
\date{\today}

\begin{document}

% --- Portada ---
{
\setbeamercolor{background canvas}{bg=AndaluciaGreen}
\begin{frame}[plain]
    \begin{center}
        \vspace{1cm}
        \textcolor{white}{\Huge \textbf{Participacion Electoral}} \\
        \vspace{0.3cm}
        \textcolor{white}{\Large Construccion del Indicador de Capital Social} \\
        \vspace{1cm}
        \begin{tikzpicture}
            \fill[VibrantOrange] (0,0) circle (0.4cm) node[white] {\textbf{V}};
            \draw[white, line width=2pt] (0,0) circle (0.4cm);
        \end{tikzpicture} \\
        \vspace{1cm}
        \textcolor{white}{\large Fuente: Barometros CIS (2016 - 2025)} \\
    \end{center}
\end{frame}
}

% --- Slide 2: El Reto ---
\begin{frame}{El Reto: Heterogeneidad en los Datos}
    \begin{columns}
        \begin{column}{0.6\textwidth}
            \begin{tcolorbox}[title=¿Por que no es directo?]
                El CIS cambia las preguntas sobre participacion electoral segun el mes y la cercania de elecciones.
            \end{tcolorbox}
            \begin{itemize}
                \item \textbf{Cuestionarios variables}: No siempre se pregunta lo mismo.
                \item \textbf{Huecos temporales}: Algunas series tienen "agujeros" (ej. 2020-2022).
                \item \textbf{Necesidad}: Una serie mensual/trimestral continua para Andalucía y España.
            \end{itemize}
        \end{column}
        \begin{column}{0.4\textwidth}
            \centering
            \begin{tikzpicture}
                \node[draw, AndaluciaGreen, line width=2pt, rounded corners] at (0,0) {Serie 1};
                \node[draw, VibrantOrange, line width=2pt, rounded corners] at (0,-1) {Serie 2};
                \node[draw, DeepBlue, line width=2pt, rounded corners] at (0,-2) {Serie 3};
                \draw[->, line width=1.5pt] (0.5,-1) -- (2,-1) node[right] {IPA27};
            \end{tikzpicture}
        \end{column}
    \end{columns}
\end{frame}

% --- Slide 3: Las 3 Fuentes de Microdatos ---
\begin{frame}{Las 3 Fuentes de "Propension al Voto"}
    \begin{small}
    \begin{itemize}
        \item \textbf{1. Escala 0-10 (Probabilidad)}: 
            \textit{"En una escala de 0 a 10... ¿con que probabilidad ira Vd. a votar?"}
            $\rightarrow$ Se calcula el \% de abstencion probable (Nota $\leq$ 4).
        \item \textbf{2. Categorias (4 Niveles)}: 
            \textit{"¿Ira Vd. a votar con toda seguridad... probablemente...?"}
            $\rightarrow$ Se agrupa "Probablemente no" + "Con seguridad no".
        \item \textbf{3. Intencion Directa}: 
            Pregunta de intencion de voto donde el entrevistado responde espontaneamente \textit{"No votaria / Me abstendria"}.
    \end{itemize}
    \end{small}
    \begin{tcolorbox}[colframe=VibrantOrange, title=Integracion]
        El notebook unifica estas tres series mediante un proceso de **Enlazado Historico** para evitar saltos artificiales.
    \end{tcolorbox}
\end{frame}

% --- Slide 4: Pipeline de Procesamiento ---
\begin{frame}{Pipeline de Calculo}
    \begin{columns}
        \begin{column}{0.5\textwidth}
            \begin{block}{1. Mapeo Dinamico}
                Identificacion automatica de variables (P16, P22, PROBVOTO) en los archivos .sav segun el numero de estudio.
            \end{block}
            \begin{block}{2. Ponderacion Regional}
                Uso de la variable \textbf{PESO} del CIS para asegurar la representatividad de cada CCAA.
            \end{block}
        \end{column}
        \begin{column}{0.5\textwidth}
            \begin{block}{3. Agregacion Mensual}
                Calculo de la media de propension al voto por mes y comunidad autonoma.
            \end{block}
            \begin{block}{4. Transformacion a "Participacion"}
                Inversion del indicador (100 - Abstencion) y escalado para el Pillar de Capital Social.
            \end{block}
        \end{column}
    \end{columns}
\end{frame}

% --- Slide 5: El Resultado para Andalucia ---
\begin{frame}{Resultado: El indicador SOC\_PAR}
    \begin{center}
        \begin{tcolorbox}[title=Serie Consolidada, colframe=AndaluciaGreen]
            Se obtiene una serie final sin saltos que refleja la **salud democratica** y el compromiso civil.
        \end{tcolorbox}
    \end{center}
    \begin{itemize}
        \item \textbf{Uso en el Pillar 4}: Capital Social.
        \item \textbf{Frecuencia}: Originalmente mensual (CIS), agregada a trimestral para el IPA27.
        \item \textbf{Interpretacion}: Valores altos (hacia 100) indican una alta movilizacion electoral potencial.
    \end{itemize}
\end{frame}

% --- Slide 6: Cierre ---
{
\setbeamercolor{background canvas}{bg=DeepBlue}
\begin{frame}[plain]
    \begin{center}
        \vspace{2cm}
        \textcolor{white}{\Huge \textbf{Conectando Datos y Sociedad}} \\
        \vspace{1.5cm}
        \textcolor{VibrantOrange}{\large Metodologia Electoral IPA27} \\
    \end{center}
\end{frame}
}

\end{document}
